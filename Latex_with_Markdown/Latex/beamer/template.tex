%文档开头定义文章类型
\documentclass[UTF8,table]{ctexbeamer}

%指定beamer的模式与主题
\mode<presentation>
{
	\usetheme{Madrid}
	%\usetheme{Boadilla}
	%\usecolortheme{default}
	\usecolortheme{orchid}
	%\usecolortheme{whale}
	%\usefonttheme{professionalfonts}
}

%使用哪些宏包
\usepackage{xcolor}

%分栏
\usepackage{multicol}

% 插入图片
\usepackage{graphicx}
\usepackage{eso-pic}
\usepackage{color}
% 指定存储图片的路径(当前目录下的figures文件夹)
\graphicspath{{figures/}}

% 设定英文字体
\usepackage{fontspec}
\setmainfont{Times New Roman}
\setsansfont{Arial}
\setmonofont{Courier New}

% 设定中文字体
\hypersetup{xetex,bookmarksnumbered=true,bookmarksopen=true,pdfborder=1,breaklinks,colorlinks,linkcolor=cyan,filecolor=black,urlcolor=blue,citecolor=green}

%去除图表标题中的figure等
\usepackage{caption}
\captionsetup{labelformat=empty,labelsep=none}

%设置行距
\usepackage{setspace}

%插入源代码
\usepackage{listings}
\lstset{
	language=perl,                  % 程序语言名称:TeX, Perl, R, sh, bash, Awk
	basicstyle=\normalsize\tt,      %\tt指monospace字体族,程序源代码使用此族字体表示更加美观
	numbers=left,                   % 行号位置(左侧)
	numberstyle=\small,             % 行号字体的字号
	stepnumber=1,                   % 行号的显示步长
	numbersep=5pt,                  % 行号与代码间距
	backgroundcolor=\color{white},  % 背景色;需要 \usepackage{color}
	showspaces=false,               % 不显示空格
	showstringspaces=false,         % 不显示代码字符串中的空格标记
	showtabs=false,                 % 不显示 TAB
	tabsize=4, 
	frame=shadowbox,                % 把代码用带有阴影的框圈起来
	captionpos=b,                   % 标题位置
	breaklines=true,                % 对过长的代码自动断行
	breakatwhitespace=false,        % 断行只在空格处
	extendedchars=false,            % 解决代码跨页时,章节标题,页眉等汉字不显示的问题
	%escapeinside={\%*}{*},         % 跳脱字符,添加注释,暂时离开 listings 
	%escapeinside=``,
	commentstyle=\color{red!50!green!50!blue!50}\tt,  %浅灰色的注释
	rulesepcolor=\color{red!20!green!20!blue!20},     %代码块边框为淡青色
	keywordstyle=\color{blue!70}\bfseries\tt,         %代码关键字的颜色为蓝色,粗体
	identifierstyle=\tt,
	stringstyle=\tt,                % 代码字符串的特殊格式
	keepspaces=true,
	breakindent=1em,
	%breakindent=22pt,
	%breakindent=4em,
	breakautoindent=true,
	flexiblecolumns=true,
	aboveskip=1em,                  %代码块边框
	xleftmargin=2em,
	xrightmargin=2em
}

%\setbeamercolor{alerted text}{fg=magenta}
\setbeamercolor{bgcolor}{fg=yellow,bg=cyan}
%\setbeamercolor{itemize/enumerate body}{fg=green}
	
% 重定义字号命令
\newcommand{\xiaochu}{\fontsize{30pt}{40pt}\selectfont}    % 小初, 1.5倍行距
\newcommand{\yihao}{\fontsize{26pt}{36pt}\selectfont}    % 一号, 1.4倍行距
\newcommand{\erhao}{\fontsize{22pt}{28pt}\selectfont}    % 二号, 1.25倍行距
\newcommand{\xiaoer}{\fontsize{18pt}{18pt}\selectfont}    % 小二, 单倍行距
\newcommand{\sanhao}{\fontsize{16pt}{24pt}\selectfont}    % 三号, 1.5倍行距
\newcommand{\xiaosan}{\fontsize{15pt}{22pt}\selectfont}    % 小三, 1.5倍行距
\newcommand{\sihao}{\fontsize{14pt}{21pt}\selectfont}    % 四号, 1.5倍行距
\newcommand{\banxiaosi}{\fontsize{13pt}{19.5pt}\selectfont}    % 半小四, 1.5倍行距
\newcommand{\xiaosi}{\fontsize{12pt}{15pt}\selectfont}    % 小四, 1.25倍行距
\newcommand{\dawuhao}{\fontsize{11pt}{11pt}\selectfont}    % 大五号, 单倍行距
\newcommand{\wuhao}{\fontsize{10.5pt}{10.5pt}\selectfont}    % 五号, 单倍行距
\newcommand{\xiaowu}{\fontsize{9pt}{9pt}\selectfont}    % 小五号, 单倍行距
\newcommand{\liuhao}{\fontsize{7.875pt}{7.875pt}\selectfont}  % 字号设置
\newcommand{\qihao}{\fontsize{5.25pt}{5.25pt}\selectfont}    % 字号设置

%设置缩进
\usepackage{indentfirst}
\setlength{\parindent}{2em}

% 使所有隐藏的文本完全透明、动态,而且动态的范围很小
\beamertemplatetransparentcovereddynamic

% 使itemize环境中变成小球,这是一种视觉效果
\beamertemplateballitem

% 为所有已编号的部分设置一个章节目录,并且编号显示成小球
\beamertemplatenumberedballsectiontoc

% 将每一页的要素的要素名设成加粗字体
\beamertemplateboldpartpage

% item逐步显示时,使已经出现的item、正在显示的item、将要出现的item呈现不同颜色
\def\hilite<#1>{
	\temporal<#1>{\color{gray}}{\color{blue}}
	{\color{blue!25}}
}


%标题页
\begin{document}
	\title{轮转报告}
	\author[Xum]{徐孟,老严}	
	\institute[WHIOV]{武汉病毒研究所\\中国科学院}
	\date{\today}
    	\logo{\includegraphics[width=0.9cm,height=0.9cm]{1.jpg}}

% 定义目录页
\AtBeginPart{
	\frame{
		\frametitle{\partpage}
		\begin{multicols}{2}
			% 如果目录过长,可以打开这个选项分两栏显示
			\tableofcontents
			% 使用这个命令自动生成目录
		\end{multicols}
	}  
}

% 在每个Section前都会加入的Frame
\AtBeginSection[]
{
	\begin{frame}<beamer>
		\frametitle{提纲}
			\begin{multicols}{2}
		\setcounter{tocdepth}{2}
		\tableofcontents[currentsection,currentsubsection]
			\end{multicols}
	\end{frame}
}
% 在每个Subsection前都会加入的Frame
\AtBeginSubsection[]
{
	\begin{frame}<beamer>
		%\begin{frame}<handout:0>
		% handout:0 表示只在手稿中出现
		\frametitle{提纲}
			\begin{multicols}{2}
		%只显示两级目录
		\setcounter{tocdepth}{2}
		\tableofcontents[currentsection,currentsubsection]
			\end{multicols}
		% 显示在目录中加亮的当前章节
	\end{frame}
}

%%%%%%%%%%%%%%%以上为基本设置%%%%%%%%%%%%%%%%%%%%

%封面
\begin{frame}
	\titlepage
\end{frame}

\begin{frame}[plain]
	\frametitle{提纲}
		\begin{multicols}{2}
	\setcounter{tocdepth}{2}
	\tableofcontents
		\end{multicols}
\end{frame}

%第一节
\section{列表}


%无序列表加叠层
\subsection{无序列表}
\begin{frame}
	\begin{itemize}
		\hilite <1> \item 无序列表一
		\hilite <2-3> \item 无序列表二
		\hilite <3> \item 无序列表三
		\hilite <4-> \item 无序列表四
		\hilite <5> \item 无序列表五
		\hilite <-6> \item 无序列表六
		\hilite <7> \item 无序列表七
		\hilite <8> \item 无序列表八
		\hilite <9> \item 无序列表九
	\end{itemize}
\end{frame}

\subsection{有序列表}
%有序列表加叠层且显示强调
\begin{frame}
	\frametitle{有序列表}
	\begin{enumerate}[<+-|alert@+>]
		\item 有序列表一
		\item 有序列表二
		\item 有序列表三
		\item 有序列表四
		\item 有序列表五
		\item 有序列表六
		\item 有序列表七
		\item 有序列表八
		\item 有序列表九
	\end{enumerate}
\end{frame}

\section{图片}
%插入图片
\begin{frame}
	\frametitle{图片}
	\begin{figure}[htbp]
		\centering
		\includegraphics[width=8cm]{2.jpg}
		\caption{Powered by}
		\label{fig:power}
	\end{figure}
\end{frame}

\section{表格}
%插入表格
\begin{frame}
\begin{table}[tb]
	\centering
	\caption{Caption here}
	\rowcolors[]{2}{blue!30}{blue!10}
	\begin{tabular}{lcc}
		\hline
		\rowcolor{blue!30}
		\textbf{column 1} & \textbf{column 2} & \textbf{column 3} \\
		\hline
		Hello & Beamer & NAN \\
		$\alpha+\beta$ & $\gamma+\eta$ & 34\% \\
		\hline
	\end{tabular}
\end{table}
\end{frame}

%block
\section{块状结构}
\subsection{内置}
\begin{frame}
	\frametitle{block, definition, example}
	\begin{block}{块}
		这是一个block。
	\end{block}
	\pause
	\begin{definition}{定义:}
		这是一个definition。
	\end{definition}
	\pause
	\begin{example}{实例:}
		这是一个example。
	\end{example}
	\pause
	\begin{alertblock}{实例:}
		这是一个alertblock。
	\end{alertblock}
\end{frame}

\subsection{自定义}
\begin{frame}
	\begin{beamercolorbox}[rounded=true,shadow=true,wd=12cm]{bgcolor}
		这是一个自定义的彩色块状结构。
	\end{beamercolorbox}
\end{frame}

\section{分栏}
\begin{frame}
	\frametitle{分栏}
	\begin{columns}
		\column{2cm}
		这是第一栏的文字;栏宽5cm。
		\column{8cm}
		\centering 杀生丸
			\begin{figure}[htbp]
				\centering
				\includegraphics[width=8cm]{2.jpg}
				\caption{Powered by}
				\label{fig:power}
			\end{figure}
	\end{columns}
\end{frame}

\section{公式}
\begin{frame}
	\frametitle{质能方程}
	\begin{equation}
		E=mc^2
		\label{emc}
	\end{equation}
\end{frame}

\begin{frame}
\end{frame}

\section{代码}
\begin{frame}[fragile]
\frametitle{R | Example}
\begin{lstlisting}[language=R,basicstyle=\footnotesize\tt]
xrange <- c(-15, 15); yrange <- c(0, 16)
plot(0, xlim=xrange, ylim=yrange, type="n")

yr <- seq(yrange[1], yrange[2], len=50)
offsetFn <- function(y) { 2 * sin(0 + y/3) }
offset <- offsetFn(yr)
leftE <- function(y) { -10 - offsetFn(y) }
rightE <- function(y) { 10 + offsetFn(y) }
xp <- c(leftE(yr), rev(rightE(yr)))
yp <- c(yr, rev(yr))
polygon(xp, yp, col="#ffeecc", border=NA)

h <- 9
xt <- seq(0, rightE(h), len=100)
yt <- log(1 + log(1 + log(xt + 1)))
yt <- yt - min(yt); yt <- h * yt/max(yt)
x <- c(leftE(h), rightE(h), rev(xt), -xt)
y <- c(h, h, rev(yt), yt)
polygon(x, y, col="red", border=NA)
\end{lstlisting}	
\end{frame}

\begin{frame}
\end{frame}


%致谢页
\section*{Acknowledgements}
\begin{frame}
	\frametitle{Powered by}
	\begin{center}
		\includegraphics[width=9cm]{power.png}
	\end{center}
\end{frame}

\begin{frame}[plain]
	\begin{spacing}{1.5}	
		\begin{center}
			\Huge{\textbf{Thanks for your attention!}}
			
			\Huge{\textit{Any questions?}}
		\end{center}
	\end{spacing}	
\end{frame}
% 文字可选:The End | Thank You for Your Attention! | Thank You for Listening!
% 文字可选:谢谢!| 感谢聆听!
\end{document}\documentclass[UTF8]{ctexbeamer} 
