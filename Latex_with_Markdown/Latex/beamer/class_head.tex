\begin{document}

%\includeonlyframes{current}

\logo{\includegraphics[height=0.08\textwidth]{whiov.jpeg}}

% 在每个Section前都会加入的Frame
\AtBeginSection[]
{
  \begin{frame}<beamer>
    %\frametitle{Outline}
    \frametitle{提纲}
    \setcounter{tocdepth}{3}
    \begin{multicols}{2}
      \tableofcontents[currentsection,currentsubsection]
      %\tableofcontents[currentsection]
    \end{multicols}
  \end{frame}
}
% 在每个Subsection前都会加入的Frame
\AtBeginSubsection[]
{
  \begin{frame}<beamer>
%%\begin{frame}<handout:0>
%% handout:0 表示只在手稿中出现
    \frametitle{提纲}
    \setcounter{tocdepth}{3}
    \begin{multicols}{2}
    \tableofcontents[currentsection,currentsubsection]
    \end{multicols}
%% 显示在目录中加亮的当前章节
  \end{frame}
}

% 为当前幻灯片设置背景
%{
%\usebackgroundtemplate{
%\vbox to \paperheight{\vfil\hbox to
%\paperwidth{\hfil\includegraphics[width=2in]{tijmu_charcoal.png}\hfil}\vfil}
%}
%\begin{frame}[plain]
%  \begin{center}
%    {\Huge 故事中的统计学\\}
%    \vspace{1cm}
%    {\LARGE 天津医科大学\\}
%    \vspace{0.2cm}
%    {\LARGE 生物医学工程与技术学院\\}
%    \vspace{1cm}
%    {\large 2017-2018学年下学期(春)\\ 公共选修课}
%  \end{center}
%\end{frame}
%}

