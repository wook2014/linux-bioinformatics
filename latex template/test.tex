\documentclass[UTF8,table]{ctexbeamer}

\usetheme{Madrid}
\usecolortheme{rose}

\usepackage{xcolor}
\usepackage{multicol}
\usepackage{graphicx}
\usepackage{eso-pic}
\usepackage{color}
% 设定英文字体
\usepackage{fontspec}
\setmainfont{Times New Roman}
\setsansfont{Arial}
\setmonofont{Courier New}


\AtBeginSection[]
{
\frame{\begin{multicols}{2}
		\tableofcontents[currentsection]
\end{multicols}}
}

\AtBeginSubsection[]
{
\frame{\begin{multicols}{2}
		\tableofcontents[currentsubsection]
\end{multicols}}	
}

%标题页
\begin{document}
	\title{轮转报告} 
	\author[Xum]{徐孟}	
	\institute[WHIOV]{中国科学院\\武汉病毒研究所}
	\date{\today}
    	\logo{\includegraphics[width=0.9cm,height=0.9cm]{2.jpg}}
\frame{\titlepage}

%目录页
\section*{目录}
\frame{\begin{multicols}{2}
	\tableofcontents
\end{multicols}}

%第一节
\section{第一节}

%第一节的第一小节
\subsection{第一小节}

%第一张幻灯片-无序列表加叠层
\begin{frame}
	\frametitle{1.1}
	\begin{itemize}
		\item 2 is prime (two divisors: 1 and 2).
		\pause
		\item 3 is prime (two divisors: 1 and 3).
		\pause
		\item 4 is not prime (\alert{three} divisors: 1, 2, and 4).
	\end{itemize}
\end{frame}

%第二张幻灯片-插入图片
\begin{frame}
	\begin{figure}[tb]
		\centering
		\includegraphics[width=0.8\textwidth]{2.jpg}
		%	\caption{杀生丸\label{fig:figure1}}
	\end{figure}	
\end{frame}

%第三张幻灯片-插入表格
\begin{frame}
\begin{table}[tb]
	\centering
%	\caption{Caption here\label{tab:tablename}}
	\rowcolors{2}{blue!30}{blue!10}
	\begin{tabular}{lcc} \hline
		\rowcolor{blue!30}
		\textbf{column 1} & \textbf{column 2} & \textbf{column 3} \\ \hline
		Hello & Beamer & NAN \\ \hline
		$\alpha+\beta$ & $\gamma+\eta$ & 34\% \\ \hline
	\end{tabular}
\end{table}
\end{frame}

%第四张幻灯片-block
\begin{frame}
	\frametitle{你喜欢\alert{吃什么}?}
	\begin{block}{你喜欢吃什么\alert{水果}}
		西瓜
	\end{block}
	\begin{block}{你喜欢吃什么肉}
		\begin{itemize}
		\item 牛肉
		\pause
		\item 猪肉
		\pause
		\item 鸡肉
	\end{itemize}
	\end{block}
\end{frame}

\subsection{第二小节}
%\tableofcontents[currentsection]
\begin{frame}
	\frametitle{1.2}
	哈哈哈
\end{frame}


\section{第二节}
\begin{frame}
	\frametitle{2}
	哈哈哈
\end{frame}

\subsection{第一小节}
%\tableofcontents[currentsection]
\begin{frame}
	\frametitle{2.1}
	哈哈哈
\end{frame}
\subsection{第二小节}
\begin{frame}
	\frametitle{2.2}
	哈哈哈
\end{frame}

\subsection{第三小节}
\begin{frame}
	\frametitle{2.3}
	哈哈哈
\end{frame}

\section{3}

\subsection{第一小节}
%\tableofcontents[currentsection]
\subsection{第二小节}
\begin{frame}
	\frametitle{用户和组 | 账户 | \alert{三大类}}
	\begin{enumerate}
		\item<1-> 根账户(根用户/超级用户账户,root,UID=0)
		\begin{itemize}
			\item<4-> 不受任何限制,能够进行任何操作,包括自我毁灭
			\item<5-> 谨慎使用,仅在必要且用于最重要的任务时使用
		\end{itemize}
		\item<2-> 系统账户(伪用户,UID:1-499/999;CentOS/Ubuntu)
		\begin{itemize}
			\item<6-> 与系统和程序服务相关
			\begin{itemize}
				\item bin、daemon、shutdown、halt等,任何Linux系统默认都有这些伪用户
				\item mail、news、games、apache、ftp、mysql及sshd等,与Linux系统的进程相关
				\item 通常不需要或无法登录系统
				\item 可以没有家目录
			\end{itemize}
			\item<6-> 由操作系统在安装过程中提供或由软件制造商提供
			\item<6-> 对系统特定组件进行操作,协助用户所需的服务或程序
			\item<7-> 不要轻易修改,否则可能会给系统带来不良影响
		\end{itemize}
		\item<3-> 用户账户(普通用户账户,UID:500/1000-65535;CentOS/Ubuntu)
		\begin{itemize}
			\item<8-> 为用户和用户组提供对系统的交互式访问
			\item<8-> 对关键系统文件和目录的访问权限是有限的
		\end{itemize}
	\end{enumerate}
\end{frame}

\subsection{第三小节}
\begin{frame}
	\frametitle{What's Still To Do?}
	\begin{block}{Answered Questions}
		How many primes are there?
	\end{block}
	\begin{block}{Open Questions}
		Is every even number the sum of two primes?
	\end{block}
\end{frame}
\subsection{第四小节}
\subsection{第五小节}

%第一页幻灯片
\begin{frame}{What Are Prime Numbers?}

	A prime number is a number that has exactly two divisors.
\end{frame}

%第二页幻灯片
\begin{frame}
\begin{figure}[tb]
	\centering
	\includegraphics[width=0.8\textwidth]{2.jpg}
%	\caption{杀生丸\label{fig:figure1}}
\end{figure}	
\end{frame}

%第三页幻灯片
\begin{frame}
\begin{table}[tb]
	\centering
	\caption{表1\label{tab:tablename}}
	\begin{tabular}{l|cc} \hline
		\textbf{column 1} & \textbf{column 2} & \textbf{column 3} \\ \hline
		Hello & Beamer & NAN \\ \hline
		$\alpha+\beta$ & $\gamma+\eta$ & 34\% \\ \hline
	\end{tabular}
\end{table}	
\end{frame}

\end{document}\documentclass[UTF8]{ctexbeamer}